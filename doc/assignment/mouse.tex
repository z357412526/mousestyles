\documentclass[11pt, oneside]{article}   	% use "amsart" instead of "article" for AMSLaTeX format
\usepackage{geometry}                		% See geometry.pdf to learn the layout options. There are lots.
\geometry{letterpaper}                   		% ... or a4paper or a5paper or ... 
%\geometry{landscape}                		% Activate for rotated page geometry
%\usepackage[parfill]{parskip}    		% Activate to begin paragraphs with an empty line rather than an indent
\usepackage{graphicx}				% Use pdf, png, jpg, or eps§ with pdflatex; use eps in DVI mode
								% TeX will automatically convert eps --> pdf in pdflatex		
\usepackage{amssymb}
\usepackage{upquote}

%-----------------------------------------------------------------------------
% Special-purpose color definitions (dark enough to print OK in black and white)
\usepackage{color}
% A few colors to replace the defaults for certain link types
\definecolor{orange}{cmyk}{0,0.4,0.8,0.2}
\definecolor{darkorange}{rgb}{.71,0.21,0.01}
\definecolor{darkgreen}{rgb}{.12,.54,.11}
%-----------------------------------------------------------------------------
% The hyperref package gives us a pdf with properly built
% internal navigation ('pdf bookmarks' for the table of contents,
% internal cross-reference links, web links for URLs, etc.)
\usepackage{hyperref}
\hypersetup{pdftex, % needed for pdflatex
  breaklinks=true, % so long urls are correctly broken across lines
  colorlinks=true,
  urlcolor=blue,
  linkcolor=darkorange,
  citecolor=darkgreen,
}

\usepackage{booktabs}


\title{Stat 222: Mouse Lifestyles}
%\date{}							% Activate to display a given date or no date

\begin{document}
\maketitle

\section{Data Description}

For this project, your primary data source will be mouse behavioral data from
the Tecott Lab at
UCSF.\footnote{\url{http://www.neuroscience.ucsf.edu/neurograd/faculty/tecott.html}}
The lab has recently developed a method for continuous high-resolution
behavioral data collection and analysis, which enables them to observe and
study the structure of spontaneous patterns of behavior (``Lifestyles'') in the
mouse \cite{tecott2003genes, tecott2004neurobehavioral, goulding2008robust,
hillar2016activestate}.  They have found that using this method: 1) reveals a
set of fundamental principles of behavioral organization that have not been
previously reported, 2) permits classification by genotype with unprecedented
accuracy, and 3) enables fine dissection of behavioral patterns.

A central goal for this project is to give you hands-on experience working,
as a member of a consulting team, on a substantial Python package.  While the
Tecott Lab will provide guidance on how to analyze the data, as a member of
the class, you will each be responsible for helping to refine the project
scope and deliverables.

\section{Your Assignment}

Working as a fictitious consulting company, Capstone Analytics, we will produce
a high quality Python package to analyze some of the new data generated in the
Tecott lab.  I (Jarrod) will act as the head of Capstone Analytics and you will
all act as employees. 

As a team, we will work with the Tecott Lab to refine the project scope and
focus.  In order to manage all of your contributions, I will require that (1)
all contributions undergo a rigorous review process, (2) all code is tested and
that those tests are included as part of our automated test suite, and (3) all
code and text components follow the best practices adopted by many open source
scientific Python packages \cite{millman2014developing}.  All project code will
be distributed under the
\href{https://en.wikipedia.org/wiki/BSD_licenses\#2-clause_license_.28.22Simplified_BSD_License.22_or_.22FreeBSD_License.22.29}{simplified BSD license}
and test data will be made publicly available.  Project documentation will
be published using \href{https://pages.github.com/}{GitHub Pages}.

The Tecott Lab has provided the following project topics:

\begin{enumerate}
\item Behavioral profile
\item Exploration and path diversity
\item Dynamics of Active State patterns
\item Ultradian Analysis
\item Application of clustering algorithms
\item Power laws and universality
\end{enumerate}

\subsection*{Timeline and logistics}

Here is the tentative schedule:

\begin{table}[h]
\centering
\begin{tabular}{@{}l|l@{}}
\toprule
\multicolumn{1}{c|}{Monday} & \multicolumn{1}{c}{Wednesday} \\
\hline
(3/7) Project introduction     & (3/9) Git workflow I \\
(3/14) Follow-up discussion    & (3/16) Git workflow II \\
\emph{\hspace{12mm} Spring break}  & \emph{\hspace{12mm} Spring break}\\
(3/28) Start final project     & (3/30) TBD\\
(4/4) Project check in         & (4/6) TBD\\
(4/11) Project check in        & (4/13) TBD\\
(4/18) Project check in        & (4/20) TBD\\
(4/25) Project check in        & (4/27) TBD\\
(5/2) Project check in         & (5/4) TBD\\
\emph{\hspace{12mm} RRR week}  & \emph{\hspace{12mm} RRR week}\\
\emph{\hspace{12mm} Final week}  & \emph{\hspace{12mm} Final week}\\
%(5/9) RRR week                 & (5/11) RRR week\\
%(5/16) Final week              & (5/18) Final week\\
\bottomrule
\end{tabular}
\end{table}

On Monday, March 7th during class, Professor Larry Tecott and Dr. Chris Hillar
will provide a general introduction to the project from the perspective of a
behavioral neuroscientist (Professor Tecott) as well as from the perspective of
a computational data scientist (Dr. Hillar).  

I will lead short follow-up discussions on Monday, March 14th and Wednesday,
March 16th during class.  I will also introduce the GitHub workflow we will be
using for this project prior to spring break.  Over spring break, you should
carefully read the articles referenced at the end of this document.  You should
also decide on the top 3 subprojects you would like to work on. While
considering subprojects, you should think about what skills you can bring to
the project as well as what skills you would like to strengthen.  We will
finalize teams and subprojects the week after spring break and then will focus
on the final project for the remainder of the semester.

\section{Project Deliverables}

There will be two main project deliverables: a Python package and a
final self-evaluation.  The entire class will be responsible for
the Python package and there will be one grade for the final project.
Each student will be required to submit a final self evaluation
and to schedule a 30 minute meeting with me during finals to discuss
your evaluations.

\subsection{Python package}

I will be responsible for creating the initial project infrastructure
on GitHub.  If there are code or infrastructure issues that the
class can't agree on, I will be responsible for making the final
decision.  However, before I intervene, you will need to carefully
think through the issue and prepare arguments for and against
any decision you wish me to make.

While you will be responsible for the majority of design decisions,
I will require that the Python package have:

\begin{itemize}
\item an automated test suite with a reasonably high test coverage,
\item a comprehensive code review process for all contributed code (using
   GitHub's pull request mechanism and continuous integration using
   TravisCI and Coveralls), and
\item extensive, high quality documentation using Sphinx.
\end{itemize}

I will maintain the official project
repository\footnote{\url{https://github.com/berkeley-stat222/mouse-lifestyles}}
and will be the primary gatekeeper (i.e., I will be the one primarily
responsible for merging all pull requests).  However, I will require that pull
requests undergo a high level of review and scrutiny before I will consider
merging them.  As a class, we will develop a code review process, which will
include (among other things) program correctness, test coverage, code
readability, and style consistency.

\subsection{Self evaluation}

The purpose of the self evaluation is to provide you with an opportunity to
reflect on your contributions and growth during the final project.
Additionally, it is your opportunity to make sure I am aware of what you
contributed to the final project.  You will also be asked to reflect on
the overall course as well as the MA program in general.

After spring break, I will provide additional details regarding the self
evaluation.  I will also ask you to track your weekly goals and progress
electronically.  You should use this weekly planning process to make
sure your final self evaluation includes numerous concrete examples.

Here are few things to keep in mind while working on your self evaluation:
\begin{enumerate}
\item Take responsibility for failures and shortcomings.
\item Show confidence, not arrogance.
\item Do not over embellish.
\item Outline constraints you faced as well as reasons performance was hampered.
\item Include your weakness, but view them as opportunities for improvement.
\item Provide feedback on your experience during the project, course, and program.
\item Stay objective.
\item Demonstrate areas of growth.
\item Highlight skills acquired.
\item Include a discussion of problem-solving abilities you used during the project.
\end{enumerate}


\bibliographystyle{plain}
\bibliography{mouse}

\end{document}
