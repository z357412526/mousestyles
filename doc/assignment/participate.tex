\documentclass[11pt, oneside]{article}   	% use "amsart" instead of "article" for AMSLaTeX format
\usepackage{geometry}                		% See geometry.pdf to learn the layout options. There are lots.
\geometry{letterpaper}                   		% ... or a4paper or a5paper or ...
%\geometry{landscape}                		% Activate for rotated page geometry
%\usepackage[parfill]{parskip}    		% Activate to begin paragraphs with an empty line rather than an indent
\usepackage{graphicx}				% Use pdf, png, jpg, or eps§ with pdflatex; use eps in DVI mode
								% TeX will automatically convert eps --> pdf in pdflatex
\usepackage{amssymb}
\usepackage{upquote}

%-----------------------------------------------------------------------------
% Special-purpose color definitions (dark enough to print OK in black and white)
\usepackage{color}
% A few colors to replace the defaults for certain link types
\definecolor{orange}{cmyk}{0,0.4,0.8,0.2}
\definecolor{darkorange}{rgb}{.71,0.21,0.01}
\definecolor{darkgreen}{rgb}{.12,.54,.11}
%-----------------------------------------------------------------------------
% The hyperref package gives us a pdf with properly built
% internal navigation ('pdf bookmarks' for the table of contents,
% internal cross-reference links, web links for URLs, etc.)
\usepackage{hyperref}
\hypersetup{pdftex, % needed for pdflatex
  breaklinks=true, % so long urls are correctly broken across lines
  colorlinks=true,
  urlcolor=blue,
  linkcolor=darkorange,
  citecolor=darkgreen,
}

\usepackage{booktabs}


\title{Stat 222: Mousestyles participation}
%\date{}							% Activate to display a given date or no date

\begin{document}
\maketitle

\section{Self evaluation}

Email me a 3-page self evaluation with \texttt{[STAT222 SELFEVAL]} in the subject line
by 5PM on Friday, May 13th.

Your self evaluation should be structured as follows:

\begin{enumerate}

\item Briefly describe your background.

\item Discuss two significant pull request that you worked on for the final project.

\item Discuss one pull request for which you did a significant part of the review.

\item Conclude by answering a few questions.

\end{enumerate}

For most of you, the project workflow was probably new.  So I would particularly
like to hear how you adapted to the process and which parts you found most
difficult to master.  Were there specific barriers you had to overcome?  How
did you do that?

\subsection{Background}

In a paragraph briefly describe your programming background.  Have you
worked on any big software projects before?  How much experience with
Git and Python did you have before this class?  Had you ever written
automated unit tests before?  Had you ever had your code peer-reviewed
before?

\subsection{Pull requests}

Choose two pull requests that you worked on for the final project.
It is OK to discuss a pull request, which was ultimately not merged
in the project repository either because it is unfinished or because
it was rejected.
One of the pull requests should involve you contributing a significant
function or piece of code.  The other pull request should involve
you working on the text of the report.  Please provide links to the
two pull requests.

For each pull request discuss the review process.  How did your code
or text contribution evolve during the review process?  Did you
uncover bugs or errors in your code and/or text as a result of the
review process or the tests?  If you cowrote the pull request, please
list your coauthors.  Did you rebase or merge from upstream master?
How did you decide whether to merge from master or rebase?
In the end, was your pull request merged?  Why or why not?
What did your pull request contribute to the overall project?

\subsection{Pull request review}

Choose one pull request for which you did a significant part of the review.
Please include a link to the pull request.

Describe what the pull request included (e.g., what functions or part of the
report did it address).  Did you find any errors in the code or text?  How
did the pull request author(s) respond to your review?  Where you able to
improve the pull request?  If so, in what way did you improve it?

\subsection{Questions}

Finally address the following questions:

\begin{enumerate}

\item How did you improve your understanding of Git and Python during the
project?  Did you do outside reading?  If so, what resources did you find
helpful?

\item How did you help organize the work that your team did?

\item What did you contribute to the team (in addition to the above pull requests)?

\item What did you learn about your project that you think was most surprising or interesting?

\item Is there anything else you think I should know?

\item What is the most valuable thing you got out of the class?

\item Which aspect of the class did you find least valuable?

\end{enumerate}

\section{General instructions}

Here are few things to keep in mind while working on your self evaluation:
\begin{enumerate}
\item Take responsibility for failures and shortcomings.
\item Do not embellish.
\item Outline constraints you faced as well as reasons performance was hampered.
\item Include your weakness, but view them as opportunities for improvement.
\item Provide feedback on your experience during the project, course, and program.
\item Stay objective.
\item Demonstrate areas of growth.
\item Highlight skills acquired.
\item Include a discussion of problem-solving abilities you used during the project.
\end{enumerate}

\end{document}
